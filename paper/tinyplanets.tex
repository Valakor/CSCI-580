\documentclass[a4paper,twoside]{article}

\usepackage{epsfig}
\usepackage{subfigure}
\usepackage{calc}
\usepackage{amssymb}
\usepackage{amstext}
\usepackage{amsmath}
\usepackage{amsthm}
\usepackage{multicol}
\usepackage{pslatex}
\usepackage{apalike}
\usepackage{graphicx}
\usepackage{algorithmic}
\usepackage{textcomp}
\usepackage{SCITEPRESS}     % Please add other packages that you may need BEFORE the SCITEPRESS.sty package.

\subfigtopskip=0pt
\subfigcapskip=0pt
\subfigbottomskip=0pt

\begin{document}

\title{Tiny Planets}

\author{\authorname{Sharzard Gustafson, Matthew Pohlmann, Ricardo Sisnett}
\affiliation{University of Southern California, Los Angeles, USA.}
\email{\{sagustaf, pohlmann, sisnetth\}@usc.edu}
}

\keywords{Procedural Generation, Graphics, Perlin Noise}

\abstract{ In this paper we present a method of procedurally generating planetary models with geography by combining multi-fractal Perlin noise and recursive icospheres. ... placement of  trees and ..., also procedurally modeled, ... A material system was added to visualize the results in the graphics engine... }

\onecolumn \maketitle \normalsize \vfill

\section{\uppercase{Introduction}}
\label{sec:introduction}
\noindent In this work we present the results of implementing a series of techniques in a bare bones graphics engine, extending its core by adding a materials system and creating an application that utilizes procedural modeling techniques to create planetoids with natural looking geographical features.

\section{\uppercase{Environment}}
\label{sec:env}

\subsection{Material System}
\label{sec:mat_sys}

\section{\uppercase{IcoSpheres}}
\label{sec:icosphere}
\noindent An icosphere is one of two of the most common ways of rendering spheres in computer graphics applications, the other being the UV sphere. An icosphere starts by taking a 20 sided polyhedron (icosahedron)  and recursively subdividing the faces to add detail. This method of generation makes the sphere's vertexes to be evenly distributed along the surface, which is a desirable trait for our use case since it lends itself better for deformation than UV spheres [REF]. Figure \ref{fig_isa} illustrates the basic idea behind this algorithm.

\begin{figure}
\caption{Representation of the icosphere subdivision algorithm}
\label{fig_isa}
\end{figure}

\section{\uppercase{Perlin Noise}}
\label{sec:pnoise}
\noindent Ken Perlin's algorithm for creating pseudo-random noise is a staple in any application trying to replicate natural occurring patterns. The algorithm is fairly simple and consists of three steps: defining an n-dimensional grid, computing the dot value of distance-gradient vectors and interpolating this values.

\begin{figure}
\caption{Textured generated by our implementation of the Perlin Noise function}
\label{fig_pn}
\end{figure}

\subsection{Multi Fractal Perlin Noise}
\label{sec:mfpnois}
Classical Perlin noise is homogeneous and isotropic, this plays against the rationale to use Perlin noise: creating natural occurring patterns.  As defined in [REF] a \textit{multifractal} is a heterogeneous fractal, this is achieved by parameterizing the dimension of the fractal function with another attribute, such as the height value of the terrain, this makes it so values closer to 0 (sea level) are smoothed out, whereas high values higher octaves have a bigger effect causing jags. The instance of the algorithm implemented for this work is described in Figure \ref{fig_mf}.

\begin{figure}
\caption{Multifractal algorithm}
\label{fig_mf}
\end{figure}

\section{\uppercase{Terrain Generation}}
\label{sec:tgeneration}

\noindent Using an instance of the multi-fractal Perlin noise, we deform vertexes as they are being created by the icosphere algorithm to create geographic-looking features. We use a \textit{ridged} version of the algorithm to produce more interesting patterns as described in [REF].

The deformation is the result of displacing the vertex by a value up to a maximum deformation in the direction of the vertex normal vector, which could be interpreted as 'up' in the terrains frame of reference, thus creating a relief.

\begin{figure}
\caption{Terrain deformation algorithm illustrated}
\label{fig_tda}
\end{figure}

\begin{figure}
\caption{Planets generated by different levels of recursion on the icospheres}
\label{fig_iterations}
\end{figure}

\section{\uppercase{Results}}
\label{sec:results}

\vfill
\end{document}

